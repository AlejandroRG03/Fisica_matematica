\documentclass{article}
\usepackage[spanish]{babel}
\decimalpoint
\usepackage{graphicx} % Required for inserting images
\usepackage[margin = 1 in]{geometry}
\usepackage{bbold}
\usepackage{amssymb}
\usepackage{amsmath}
\usepackage{ytableau}
\usepackage{physics}

\title{Fisica matematica}
\author{Alejandro Rodríguez González \\ Lorena Rodríguez Lamas}
\date{January 2026}

\begin{document}

\maketitle

\section{Introducción}

El grupo $SU(N)$ es el conjunto por matrices unitarias $N\times N$ con determinante unidad, que tiene por ley de composición el producto convencional de matrices. Es decir:
\begin{equation}
    U U ^\dagger = U^\dagger U = \mathbb{1}_{N\times N}, \quad \det(U) = 1 \quad \forall U\in SU(N)
\end{equation}
Un elemento del grupo $SU(N)$ se puede parametrizar de forma genérica de la siguiente forma:
\begin{equation}
    U(\theta) = e^{i\theta^a T^a}, \quad \theta^a \in \mathbb{R}
\end{equation}
Donde $T^a$ son los generadores del álgebra de Lie $\mathfrak{su}(N)$. Dichos generadores deben ser hermíticos y de traza nula para respetar la definición del grupo. Estas restricciones limitan el número de generadores independientes del grupo a un total de $N^2 -1$.

El álgebra $\mathfrak{su}(N)$, para una cierta representación de los generadores $T^a$ que llamaremos $D$, es generada a partir de las relaciones de conmutación:
\begin{equation}
    [T^a, T^b] = i f^{abc}T^c, \quad f^{abc} \in \mathbb{R}
\end{equation}
De esta ecuación es inmediato ver que los operadores $-T^{a*}$ cumplen las mismas relaciones de conmutación, es decir:
\begin{equation}
    [-T^{a*}, -T^{b*}] = i f^{abc}(-T^{c*})
\end{equation}
Esto se conoce como representación conjugada de $D$, usualmente denotada $\bar{D}$.

El estudio del grupo $SU(N)$ tiene especial interés en física. Por ejemplo, sabemos que el grupo $SU(2)$ describe el espín. Así, la función de onda de describe el espín de un electrón vive en la representación fundamental de $SU(2)$, que tiene dimensión $2$. Esto simplemente nos dice que tenemos $2$ estados posibles de espín, $\ket{\uparrow}, \ket{\downarrow}$. En añadido, la física cuántica nos dice que si tenemos dos electrones en estados $\ket{\psi_1}, \ket{\psi_2}$, la función de onda que describe el sistema global se obtiene simplemente como el producto tensorial de ambos estados, es decir:
\begin{equation}
    \ket{\Psi} = \ket{\psi_1} \otimes \ket{\psi_2} = \{\ket{\uparrow\uparrow}, \ket{\downarrow\downarrow}, \ket{\uparrow\downarrow}, \ket{\downarrow\uparrow}\}
\end{equation}
Sin embargo, esta multiplicación bruta no nos da los estados físicos del sistema. Esto es debido a que los electrones son partículas indistinguibles, por lo que la función de ondas debe ser simétrica/antisimétrica bajo el intercambio de 2 electrones, algo que los dos estados con espines cruzados no cumplen. Físicamente, sabemos que el sistema puede estar en un triplete de espín ($S=1$), o en un singlete de espín ($S=0$), de forma que podemos escribir los estados físicos como:
\begin{align}
    \text{Triplete} &\equiv \left\{\ket{\uparrow\uparrow}, \ket{\downarrow\downarrow}, \frac{1}{\sqrt{2}}\left(\ket{\uparrow\downarrow}+\ket{{\downarrow\uparrow}}\right)\right\}
    \\
    \text{Singlete} &\equiv \left\{ \frac{1}{\sqrt{2}}\left(\ket{\uparrow\downarrow}-\ket{{\downarrow\uparrow}}\right)\right\}
\end{align}
Es decir, pasamos de 4 estados desordenados a dos agrupaciones con propiedades físicas bien definidas; una con 3 elementos simétricos, o de dimensión 3; y otra con 1 elemento antisimétrico, o de dimensión 1. De forma efectiva hemos determinado que:
\begin{equation}
    2\otimes 2 = 3_S\oplus 1_A
\end{equation}
Esto se conoce como la \textbf{descomposición de Clebsch-Gordan.} En general, la descomposición consiste en eexpresar el producto tensorial de un cierto número de representaciones $R_1\otimes R_2 \otimes R_3 \cdots$ en una suma de representaciones irreducibles $r_1\oplus r_2\oplus r_3\cdots$. \textbf{Dichas representaciones irreducibles se corresponden con los estados físicos del sistema}.

Entonces, hemos visto que es de particular interés obtener descomposición de Clebsch-Gordan del sistema a estudiar. La descomposición en representaciones irreducibles y sus propiedades de simetría se puede sistematizar mediante el formalismo de los tableros de Young.


\section{Reglas de construcción de los tableros}
Un tablero de Young es un esquema bidimensional de cajas organizadas en columnas y filas. Existen dos restricciones:
\begin{itemize}
    \item El número de cajas en una fila debe ser igual o menor al número de cajas de la fila anterior.
    \item El número de cajas en una columna debe ser menor o igual que $N$. 
\end{itemize}
Por ejemplo para $SU(3)$
\begin{center}
    \begin{ytableau}
        {} & {} & {} \\
        {} & {}
    \end{ytableau}
    \hspace{0.5cm}
    \begin{ytableau}
        {} & {} & {} \\
        {} & {} & {} & {}
    \end{ytableau}
    \hspace{0.5 cm}
    \begin{ytableau}
        {} & {} \\
        {} \\
        {} \\
        {}
    \end{ytableau}
\end{center}
\hspace{4.8cm} Permitido \hspace{0.5 cm} No permitido \hspace{0.5 cm} No permitido

Un diagrama puede ser asociado con una representación irreducible del grupo $SU(N)$. La \textbf{representación }
\section{Descomposición de un producto tensorial}
Para llevar a cabo la descomposición Clebsh-Gordan de un producto tensorial de dos tableros de Young, llamémoslos $A$ y $B$, procedemos de la siguiente forma:
\begin{enumerate}
    \item Etiquetamos las filas del tablero $B$ con distintas letras: la primera fila con $a$, la segunda con $b$, la tercera con $c$, y así sucesivamente.
    \item Empezamos con las $a$ y las colocamos en el diagrama $A$ con la condición de que no puede haber dos en la misma columna.
    \item Seguimos con las $b$ y las colocamos en el diagrama anterior (tras colocar las $a$) respetando la condición de no más de dos $b$ en la misma columna, y además respetando que tiene que haber siempre antes al menos una $a$ si leemos el tablero de derecha a izquierda y de arriba a abajo.
    \end{enumerate}
Así se sucedería con las $c$, $d$... siempre respetando que leyendo de derecha a izquierda y de arriba a abajo, el número de $a$ tiene que ser mayor o igual que el de $b$; que es mayor o igual que el de $c$; que es mayor o igual que el de $d$, etc.
Para familiarizarnos con estas reglas empezamos con un par de ejemplos sencillos.
\subsection{Caso $SU(2): 2 \otimes 2$}
El producto tensorial de dos partículas de espín 1/2.
\begin{center}
\begin{ytableau}
    {}
\end{ytableau}
\hspace{0.1cm}
$\otimes$
\hspace{0.1cm}
\begin{ytableau}
    a
\end{ytableau}
\hspace{0.1cm}
$=$
\hspace{0.1cm}
\begin{ytableau}
    {} & a
\end{ytableau}
\hspace{0.1cm}
$\oplus$
\hspace{0.1cm}
\begin{ytableau}
    {} \\
    a
\end{ytableau}
\end{center}
La representación reducible $2 \otimes 2$ se ha reducido a la suma directa de dos representaciones irreducibles $3 \oplus 1$.  
\subsection{Caso $SU(3): 8 \otimes 8$}
\begin{center}
\ydiagram{2,1}
\hspace{0.5 cm}
$\otimes$
\hspace{0.5 cm}
\begin{ytableau}
    a & a \\
    b
\end{ytableau}
\end{center}
Primero adjudicamos las $a$ recordando que no se pueden tener más de una por columna. De forma que tenemos
\begin{center}
 \begin{ytableau}
    {} & {} & a \\
    a
\end{ytableau}
\hspace{0.5 cm}
\begin{ytableau}
    {} & {} & a \\
    {} & a
\end{ytableau}
\hspace{0.5 cm}
\begin{ytableau}
    {} & {} & a\\
    {} \\
    a
\end{ytableau}
\hspace{0.5 cm}
\begin{ytableau}
    {} & {}\\
    {} & a \\
    a
\end{ytableau}
\end{center}
Ahora adjudicamos la $b$, teniendo en cuenta además que si leemos el diagrama de derecha a izquierda, y de arriba a abajo, en cualquier punto de la lectura tenemos que tener más $a$ que $b$. Es decir
\begin{center}
    \begin{ytableau}
        {} & {} & a & b \\
        a
    \end{ytableau}
    \hspace{0.5 cm}
    \begin{ytableau}
        {} & {} & a \\
        a & b
    \end{ytableau}
\end{center}
El primer diagrama no estaría permitido ya que empezaríamos leyendo una $b$, sin embargo el segundo sí que lo estaría ya que empezamos leyendo una $a$, seguida de la $b$. 

Con esto en mente, vamos diagrama a diagrama.
\begin{center}
\begin{ytableau}
  {} & {} & a \\
  {} & a
\end{ytableau}
\hspace{0.6cm}
:
\hspace{0.6cm}
\begin{ytableau}
  {} & {} & a \\
  {} & a & b
\end{ytableau}
\hspace{0.6cm}
\begin{ytableau}
  {} & {} & a \\
  {} & a \\
  b
\end{ytableau}

\vspace{0.2cm}
\end{center}
\hspace{8.4cm} (1) \hspace{1.8cm} (2)
\vspace{0.2cm}
\begin{center}
    \begin{ytableau}
        {} & {} \\
        {} & a \\
        a   
    \end{ytableau}
    \hspace{0.2 cm}
    : 
    \hspace{0.3 cm}
    \begin{ytableau}
        {} & {} \\
        {} & a  \\
        a & b
    \end{ytableau}
\end{center}
\vspace{0.2cm}
\hspace{9 cm} (3)
\vspace{0.2cm}
\begin{center}
    \begin{ytableau}
        {} & {} & a \\
        {} \\
        a
    \end{ytableau}
    \hspace{0.2cm}
    :
    \hspace{0.3cm}
    \begin{ytableau}
        {} & {} & a \\
        {} & b \\
        a
    \end{ytableau}
\end{center}
\vspace{0.2cm}
\hspace{9.3 cm} (4)
\vspace{0.2cm}
\begin{center}
    \begin{ytableau}
        {} & {} & a & a \\
        {}
    \end{ytableau}
    \hspace{0.2 cm}
    :
    \hspace{0.3 cm}
    \begin{ytableau}
        {} & {} & a & a \\
        {} & b
    \end{ytableau}
    \hspace{0.5cm}
    \begin{ytableau}
        {} & {} & a & a \\
        {} \\
        b
    \end{ytableau}
\end{center}
\vspace{0.2cm}
\hspace{8.1cm} (5) \hspace{2.4cm} (6)
\vspace{0.2cm}

Ahora podemos calcular la dimensión de cada diagrama
\begin{enumerate}
    \item $D = \frac{F}{H} =\frac{3\times 4 \times 5 \times 2 \times 3 \times 4}{3 \times 4 \times 2 \times 3 \times 1 \times 2} = 10$ \hspace{1.5cm} $(p, q) = (0,3) \implies \text{Representación conjugada} \implies \bar{10}$
    \item $D = \frac{F}{H} =\frac{3\times 4 \times 5 \times 2 \times 3 \times 1}{1 \times 3 \times 5 \times 1 \times 3 \times 1} = 8$ \hspace{1.5cm} $(p, q) = (1,1) \implies \text{Representación real} \implies 8$
    \item $D = \frac{F}{H} =\frac{3\times 4 \times 2 \times 3 \times 1 \times 2}{1 \times 2 \times 3 \times 2 \times 3 \times 4} = 1$ \hspace{1.5cm} $(p, q) = (0,0) \implies \text{Representación real} \implies 1$
    \item $D = \frac{F}{H} =\frac{3\times 4 \times 5 \times 2 \times 3 \times 1}{1 \times 3 \times 5 \times 1 \times 3 \times 1} = 8$ \hspace{1.5cm} $(p, q) = (1,1) \implies \text{Representación real} \implies 8$
    \item $D = \frac{F}{H} =\frac{3\times 4 \times 5 \times 6 \times 2 \times 3}{2 \times 5 \times 1 \times 4 \times 2 \times 1} = 27$ \hspace{1.5cm} $(p, q) = (2,2) \implies \text{Representación real} \implies 27$
    \item $D = \frac{F}{H} =\frac{3\times 4 \times 5 \times 6 \times 2 \times 1}{1 \times 2 \times 6 \times 3 \times 2 \times 1} = 10$ \hspace{1.5cm} $(p, q) = (3,0) \implies \text{Representación fundamental} \implies 10$
\end{enumerate}
Así que la descomposición es 
\begin{center}
    $\boxed{ 8 \otimes 8 = 27 \oplus 10 \oplus \bar{10} \oplus 8 \oplus 8 \oplus 1}$
\end{center}
\end{document}
